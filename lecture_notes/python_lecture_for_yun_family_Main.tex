\documentclass{article}

\usepackage{fullpage}

\title{Lecture Notes: Python Lecture for Yun's Family}
\author{Sunghee Yun}


\begin{document}
\maketitle

\tableofcontents

\newpage
\section{Introduction}

\subsection{Why learn computer programming?}

The following quote from a book by Briggs
well explains why we should learn computer programming.

\begin{quote}

Programming fosters creatively, reasoning, and problem solving.
The programmer gets the opportunity to create something from nothing, use logic to turn programming constructs into
a form that a computer can run, and, when things don't work quite as well as expected,
use problem solving to figure out what has gone wrong.
Programming is a fun, sometimes challenging
(and  occasionally frustrating) activity,
and the skills learned from it can be useful both in school and at work
\ldots\
even if your career has nothing to do with computers.

And, if nothing else, programming is a great way to spend an afternoon when the weather outside is dreary.

-- Jason R. Briggs, Python for Kids
\end{quote}

\nocite{JR:12}


\subsection{Why Python?}

Again here I quote some sentensces from the same book instead of my own explanation.

\begin{quote}
Python is an easy-to-learn programming language that has some really useful features for a beginning programmer.
The code is quite easy to read when compared to other programming languages,
and it has an interactive shell into which you can enter your programs and see them run.
In addition to its simple language structure and an interactive shell with which to experiment.
Python has some features that greatly augment the learning process
and allow you to put together simple animations for creating your own games.
One is the {\tt turtle} module, inspired by Turtle graphics
(used by the Logo programming language back in the 1960s)
and designed for educational use.
Another is the {\tt tkinter} module,
and interface for the Tk GUI toolkit,
which provides a simple way to create programs with slightly more advanced graphics and animation.

-- Jason R. Briggs, Python for Kids
\end{quote}


\subsection{Background}

This lecture series started by Ghayoung's suggestion that I teach Beth how to code.
I'm not sure whether I suggested it or she said it herself, 
but we also agreed that Ghayoung would join us, too.

I had believed that it would be a bad idea to teach something to my family members,
and that's why I'd never thought of this possibility.
However, I've changed my mind after a few lectures that I gave them regarding Python coding.
And here's why.

First and foremost, they had almost no problem understanding the concepts I explained to them.
Also their attitude was very good. They tried their best to understand and follow my lead
and they actually did great in absorbing the lecture contents and doing coding practice.

Secondly, I've realized that it is more valuable than anything else in the world for me to
spend my precious time and energy to convey my knowledge and experience to the people I love and care most.
I've done many talks, seminars, and lectures in my life, but none of them gave me the satisfaction
that I feel now when seeing them improving and growing on coding.

In short, unlike my long-lasting conviction that it's never a good idea to teach something to my family members,
it's turned out that it is indeed possible and it is extremely pleasing experience to do the lecture to my loved ones.
Then I realized that I've already taught many things to them including
all those math tricks and insight I tried to show Beth and numerous other concepts and stories I've told her, 
and all those (boring) conversation that I shared with Ghayoung about various topics
such as math, science, engineering, optimization, and machine learning.
The fact is that I have been doing sort of teaching already, and this is just one of those examples
with difference that, this time, we have official schedules and formats.

So I'm planning to keep this lecture series as long as I can or until they can learn themselves without my help.


\subsection{Purpose of this lecture series}

XXX

\newpage
\section{Python Basics}

\subsection{Python Dictionary}
\subsubsection{Dictionary Manipulation in Python}
\paragraph{Overview}
\begin{itemize}
	\item A dictionary is a collection of key-value pairs. 

	\item A dictionary is a set of key:value pairs. 

	\item All keys in a dictionary must be unique.

	\item In a dictionary, a key and its value are separated by a colon.

	\item The key, value pairs are separated with commas.

	\item The key \& value pairs are listed between curly brackets ``\{ \}''

	\item We query the dictionary using square brackets ``[\ ]''
\end{itemize}

\paragraph{Dictionary Manipulation}


\begin{itemize}
	\item Dictionaries are useful whenever you have to items that you wish to link together,
	and for example storing results for quick lookup.

	\item Create an empty dictionary
	\begin{verbatim}
months = {}
	\end{verbatim}

	\item Create a dictionary with some pairs (Note: Each key must be unique)

	\begin{verbatim}
months = { 1 : "January", 
    2 : "February", 
    3 : "March", 
    4 : "April", 
    5 : "May", 
    6 : "June", 
    7 : "July",
    8 : "August",
    9 : "September", 
    10 : "October", 
    11 : "November",
    12 : "December" } 
	\end{verbatim}


	\item months ``1--12'' are keys and ``January-December'' are the values

	\item Print all keys

	\begin{verbatim}
print("The dictionary contains the following keys: ", months.keys())
	\end{verbatim}
Output:
	\begin{verbatim}
The dictionary contains the following keys:  [1, 2, 3, 4, 5, 6, 7, 8, 9, 10, 11, 12]
	\end{verbatim}

	\item Accessing

	To get a value out of a dictionary, you must supply its key, you cannot provide
	the value and get the key
	\begin{verbatim}
whichMonth = months[1]
print(whichMonth)
	\end{verbatim}
Output:
	\begin{verbatim}
January
	\end{verbatim}

	\item To delete an element from a dictionary, use {\tt del}


	\begin{verbatim}
del(months[5])
print(months.keys())
	\end{verbatim}
Output:
	\begin{verbatim}
[1, 2, 3, 4, 6, 7, 8, 9, 10, 11, 12]
	\end{verbatim}

\item To add a new element to a dictionary, assign a value to a new key

	\begin{verbatim}
months[5] = "May"
print(months.keys())
	\end{verbatim}
Output:
	\begin{verbatim}
[1, 2, 3, 4, 5, 6, 7, 8, 9, 10, 11, 12]
	\end{verbatim}

\item To update an element of a dictionary, assign a new value to its key
	\begin{verbatim}
months[1] = "Jan"
print(months)
	\end{verbatim}
Output:
	\begin{verbatim}
{1: 'Jan', 2: 'February', 3: 'March', 4: 'April', 5... }
	\end{verbatim}

\item Sorting

	\begin{verbatim}
sortedkeys = sorted(months.keys())
print(sortedkeys)
	\end{verbatim}
Output:
	\begin{verbatim}
[1, 2, 3, 4, 5, 6, 7, 8, 9, 10, 11, 12]
	\end{verbatim}

\item Dictionaries and Loops
	\begin{itemize}
	\item Iterating over keys
	\begin{verbatim}
for key in months:
    print(key, months[key])
	\end{verbatim}
Output:
	\begin{verbatim}
1 January
2 February
3 March
4 April
5 May
6 June
7 July
8 August
9 September
10 October
11 November
12 December
	\end{verbatim}
	\item Iterating over (key, value) pairs
	\begin{verbatim}
for key, value in months.items():
    print(key, value)

print "The entries in the dictionary are:"
for item in months.keys():
    print("months[ ", item, " ] = ", months[item])
	\end{verbatim}
	\end{itemize}
\item Combining List and Dictionary

Example of a list of dictionaries
	\begin{verbatim}
customers = [
    {"uid":1,"name":"John"},
    {"uid":2,"name":"Smith"},
    {"uid":3,"name":"Andersson"},
]
print(customers)
	\end{verbatim}
Output:
	\begin{verbatim}
[{'uid': 1, 'name': 'John'}, {'uid': 2, 'name': 'Smith'}, {'uid': 3, 'name': 'Andersson'}]
	\end{verbatim}

\item Print the uid and name of each customer
	\begin{verbatim}
for x in customer:
    print(x["uid"], x["name"])
	\end{verbatim}
Output:
	\begin{verbatim}
1 John
2 Smith
3 Andersson
	\end{verbatim}

\item Modify an entry

This will change the name of customer 2 from Smith to Charlie
	\begin{verbatim}
customers[2]["name"]="charlie"
print(customers)
	\end{verbatim}
Output:
	\begin{verbatim}
[{'uid': 1, 'name': 'John'}, {'uid': 2, 'name': 'Smith'}, {'uid': 3, 'name': 'charlie'}]
	\end{verbatim}
\item Add a new field to each entry
	\begin{verbatim}
for x in customers:
    x["password"]="123456" # any initial value

print(customers)
	\end{verbatim}
Output:
	\begin{verbatim}
[{'password': '123456', 'uid': 1, 'name': 'John'}, {'password': '123456', 'uid': 2,
    'name': 'Smith'}, {'password': '123456', 'uid': 3, 'name': 'Andersson'}]
	\end{verbatim}

\item Delete a field
	\begin{verbatim}
del customers[1]
print(customers)
	\end{verbatim}
Output:
	\begin{verbatim}
[{'uid': 1, 'name': 'John'}, {'uid': 3, 'name': 'Andersson'}]
	\end{verbatim}

\item Delete all fields

This will delete id field of each entry.
	\begin{verbatim}
for x in customers:
    del x["id"]
	\end{verbatim}
Output:
	\begin{verbatim}
[{'name': 'John'}, {'name': 'Smith'}, {'name': 'Andersson'}]
	\end{verbatim}
\end{itemize}

\newpage
\bibliography{mybib}{}
\bibliographystyle{plain}


\end{document}
